%%%%%%%%%%%%%%%%%%%%%%%%%%
% Theorem prover Maude
%%%%%%%%%%%%%%%%%%%%%%%%%%
%Template for LaTeX outputs

\documentclass[a4]{article}
\usepackage{hyperref}
\usepackage{../commands}

\title{System G3C}
\begin{document}
\maketitle

\tableofcontents
\input{weakeningL}
\newpage
\input{weakeningR}
\newpage
\input{height}
\newpage
\input{inv}
\newpage
\input{contractionL}
\newpage
\input{contractionR}
\newpage
\input{id-exp}
\newpage
\input{cut}
\newpage
\input{admA}
\newpage
\input{admAA}
\textbf{Note:} This rule does not preserve the height of the derivation but it produces proofs of at most $s(n)$ steps. See the specification in \texttt{prop-and-A.maude}. 

\end{document}
